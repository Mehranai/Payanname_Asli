% !TeX root=../main.tex
\chapter{نتیجه‌گیری و کارهای آتی}
%\thispagestyle{empty} 
\section{مقدمه}
تشخیص فعالیت انسان در تک تصویر به عنوان یکی از حوزه‌های پررنگ و مهم در زمینه هوش مصنوعی و بینایی ماشین مطرح است. این حوزه تحقیقاتی به تحلیل و تفسیر عملکرد انسان در محیط‌های مختلف و در زمان واقعی می‌پردازد. از آنجایی که فعالیت‌های انسانی حاصل ترکیبی پیچیده از حرکات بدنی و ارتباطات انسانی است، تشخیص این فعالیت‌ها در تک تصویر یک چالش فنی و معمولاً پیچیده است.

\section{پیشنهادات برای کارهای آتی}
برای استخراج ویژگی تصویر اغلب از ResNet استفاده می‌شود که برای نشان دادن شرایط یکسان است. درحالی که با پیشرفت روش های استخراج ویژگی و مدل های پیشرفته می‌توان از روش‌های به روزتری نیز استفاده کرد.

همچنین مدل فقط یک تصویر به عنوان ورودی جهت آموزش دریافت می‌کند که این روند آموزش را کند می‌کند. می‌توان از تکه تکه کردن تصویر استفاده کرد تا یک دسته تصویر از یک تصویر ورودی برای مدل فرستاده شود.

پیشنهاد دیگر هم وابسته نکردن مدل به ناحیه های انسان ، اشیاء و ژست است و به صورت مستقل خود مدل توانایی تشخیص این ناحیه ها را در یک رویکرد توجه داشته باشد. یعنی توجه کند که کدام ناحیه ها می‌تواند درتشخیص کمک کننده باشد و تمرکز روی آن نواحی داشته باشد.


