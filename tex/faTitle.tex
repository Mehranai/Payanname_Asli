% !TeX root=../main.tex
% در این فایل، عنوان پایان‌نامه، مشخصات خود، متن تقدیمی‌، ستایش، سپاس‌گزاری و چکیده پایان‌نامه را به فارسی، وارد کنید.
% توجه داشته باشید که جدول حاوی مشخصات پروژه/پایان‌نامه/رساله و همچنین، مشخصات داخل آن، به طور خودکار، درج می‌شود.
%%%%%%%%%%%%%%%%%%%%%%%%%%%%%%%%%%%%
% دانشگاه خود را وارد کنید
\university{دانشگاه تبریز}
% دانشکده، آموزشکده و یا پژوهشکده  خود را وارد کنید
\faculty{دانشکدهٔ مهندسی برق و کامپیوتر}
% گروه آموزشی خود را وارد کنید (در صورت نیاز)
\department{گروه مهندسی کامپیوتر}
% رشته تحصیلی خود را وارد کنید
\subject{مهندسی کامپیوتر}
% گرایش خود را وارد کنید
\field{معماری سیستم های کامپیوتری}
% عنوان پایان‌نامه را وارد کنید
\title{شبکه ارتباط سنج انسان-اشیاء و انسان-ژست برای تشخیص فعالیت انسان در تک تصویر }
% نام استاد(ان) راهنما را وارد کنید
\firstsupervisor{دکتر عبدالحمید معلمی خیاوی}
\firstsupervisorrank{استادیار}
%\secondsupervisor{دکتر راهنمای دوم}
%\secondsupervisorrank{استادیار}
% نام استاد(دان) مشاور را وارد کنید. چنانچه استاد مشاور ندارید، دستورات پایین را غیرفعال کنید.
\firstadvisor{دکتر علیرضا سخندان سرخابی}
\firstadvisorrank{استادیار}
%\secondadvisor{دکتر مشاور دوم}
% نام داوران داخلی و خارجی خود را وارد نمایید.
\internaljudge{دکتر داور داخلی}
\internaljudgerank{دانشیار}
\externaljudge{دکتر داور خارجی}
\externaljudgerank{دانشیار}
\externaljudgeuniversity{دانشگاه داور خارجی}
% نام نماینده کمیته تحصیلات تکمیلی در دانشکده \ گروه
\graduatedeputy{دکتر نماینده}
\graduatedeputyrank{دانشیار}
% نام دانشجو را وارد کنید
\name{مهران}
% نام خانوادگی دانشجو را وارد کنید
\surname{شاه محمدی}
% شماره دانشجویی دانشجو را وارد کنید
\studentID{999482103}
% تاریخ پایان‌نامه را وارد کنید
\thesisdate{بهمن 1402}
% به صورت پیش‌فرض برای پایان‌نامه‌های کارشناسی تا دکترا به ترتیب از عبارات «پروژه»، «پایان‌نامه» و «رساله» استفاده می‌شود؛ اگر  نمی‌پسندید هر عنوانی را که مایلید در دستور زیر قرار داده و آنرا از حالت توضیح خارج کنید.
%\projectLabel{پایان‌نامه}

% به صورت پیش‌فرض برای عناوین مقاطع تحصیلی کارشناسی تا دکترا به ترتیب از عبارت «کارشناسی»، «کارشناسی ارشد» و «دکتری» استفاده می‌شود؛ اگر نمی‌پسندید هر عنوانی را که مایلید در دستور زیر قرار داده و آنرا از حالت توضیح خارج کنید.
%\degree{}
%%%%%%%%%%%%%%%%%%%%%%%%%%%%%%%%%%%%%%%%%%%%%%%%%%%%
%% پایان‌نامه خود را تقدیم کنید! %%
\dedication
{
{\Large تقدیم به:}\\
\begin{flushleft}{
	\huge
	پدر و مادرم
}
\end{flushleft}
}
%% متن قدردانی %%
%% ترجیحا با توجه به ذوق و سلیقه خود متن قدردانی را تغییر دهید.
\acknowledgement{
سپاس خداوندگار حکیم را که با لطف بی‌کران خود، آدمی را به زیور عقل آراست.

در آغاز وظیفه‌  خود  می‌دانم از زحمات بی‌دریغ اساتید  راهنمای خود،  جناب آقای دکتر \textbf{عبدالحمید معلمی خیاوی} صمیمانه تشکر و  قدردانی کنم که در طول انجام این پایان‌نامه با نهایت صبوری همواره راهنما و مشوق من بودند و قطعاً بدون راهنمایی‌های ارزنده‌ ایشان، این مجموعه به انجام نمی‌رسید.

از جناب آقای دکتر \textbf{علیرضا سخندان} که  زحمت مشاوره‌، بازبینی و تصحیح این پایان‌نامه را تقبل فرمودند کمال امتنان را دارم.

و در پایان، بوسه می‌زنم بر دستان خداوندگاران مهر و مهربانی، پدر و مادر عزیزم و بعد از خدا، ستایش می‌کنم وجود مقدس‌شان را و تشکر می‌کنم از خانواده عزیزم به پاس عاطفه سرشار و گرمای امیدبخش وجودشان، که بهترین پشتیبان من بودند.
}
%%%%%%%%%%%%%%%%%%%%%%%%%%%%%%%%%%%%
%چکیده پایان‌نامه را وارد کنید
\fa-abstract{
به دلیل عدم وجود اطلاعات زمانی در تک تصویر، استفاده از روشهای متداول پردازش ویدیو برای شناسایی و تشخیص نوع فعالیت انسانی امکان پذیر نمی‌باشد و در نتیجه تشخیص این اطلاعات در تک تصویر با چالشهای فراوانی روبرو است. برای جبران عدم وجود اطلاعات زمانی،تمرکز روی اطلاعات جانبی تصویر و پس‌زمینه کارآمد است. از تک تصویر می‌توان اطلاعاتی ازجمله شکل ظاهری انسان، ژست و اشیاء مرتبط با انسان را استخراج کرد. شکل ظاهری انسان، ژست و اشیاء مرتبط با انسان شامل اطلاعات مهمی از فعالیت انسان است، امروزه جهت استخراج این ویژگی‌ها از شبکه های عصبی عمیق  استفاده می‌شود. همچنین شبکه دیگری به نام ترنسفرمر  مطرح است که درابتدا درکاربرد پردازش متن استفاده می‌شد ولی امروزه در زمینه بینایی کامپیوتر  نیز بسیار مورد استفاده قرارمی‌گیرد و درحوزه تشخیص فعالیت انسان نیز سودمند بوده است. دراین پایان نامه با استفاده از شبکه‌های عصبی عمیق،  ابتدا ویژگی‌های مهم تصویر استخراج می‌شود، سپس این ویژگی‌ها به کمک شبکه ترنسفرمر با یکدیگر ترکیب شده تا یک ارتباط‌سنجی انجام شود و رابطه بین انسان و ژست، و همچنین رابطه انسان و اشیاء مرتبط مشخص شود و شبکه نسبت به این نتایج ارتباط‌سنجی شده، فعالیت را تخمین بزند. برای ترکیب ژست با انسان از توصیف کننده زاویه بدن جهت استخراج ویژگی ژست استفاده شده است. دو مجموعه داده %
\lr{VOC‌ 2012}
 و 
\lr{Stanford40}
 در زمینه فعالیت انسان برای ارزیابی کارایی مدل و آموزش شبکه مورد استفاده قرار می‌گیرد. نتیجه مدل  پیشنهادی توصیف کننده زاویه بدن در ترکیب با خروجی بخش ارتباط‌سنج روی مجموعه داده %
\lr{VOC‌ 2012}
 دقت %
 \lr{92.33\%}
را بدست آورده است.
}
% کلمات کلیدی پایان‌نامه را وارد کنید
\keywords{تشخیص فعالیت انسان در تک تصویر، یادگیری عمیق، مکانیزم توجه، تخمین ژست، شناسایی اشیاء}
% انتهای وارد کردن فیلد‌ها
%%%%%%%%%%%%%%%%%%%%%%%%%%%%%%%%%%%%%%%%%%%%%%%%%%%%%%
